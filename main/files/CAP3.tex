\chapter{Proposta de solução}

Este capítulo detalha a execução da proposta, abordando a concepção, desenvolvimento e 
implantação do sistema embarcado Android, denominado ``BURI''. Ao longo do capítulo, será apresentada uma explicação de todos os 
módulos e técnicas que compõem a solução, desde as etapas iniciais de pesquisa e levantamento de requisitos, passando pela modelagem da arquitetura do sistema, 
até o desenvolvimento e validação do protótipo final. Portanto, cada tópico apresentado a seguir trata-se de uma fase da metodologia 
explicada na Seção \ref{cap1Metodologia}. O desenvolvimento deste projeto envolveu a aplicação de técnicas de engenharia de software e hardware, contemplando desafios 
como a integração eficiente entre os componentes do sistema, as várias formas de transmissão dos dados, e a criação de uma interface intuitiva para o aplicativo Android.

\section{Pesquisa}

Na pesquisa, o objetivo principal é encontrar trabalhos relacionados e entender as consequências fisiológicas causadas pela 
inalação de monóxido de carbono (CO). Portanto, o texto base para a explicação científica sobre o composto foi retirado 
do artigo ``Carbon monoxide poisoning: a review for clinicians'' \cite{carbon-monoxide-poisoning-varon}, pois o texto possui 
uma explicação técnica e didática de como ocorre esse envenenamento, consequências no sistema neurológico e cardiovascular, fontes de 
contaminação e métodos de tratamento utilizados na medicina.

Sobre o desenvolvimento no AOSP, o trabalho \textit{Sumaúma} \cite{ufam-aosp-sumama} tem importância central no desenvolvimento de 
soluções nativas para o Android, uma vez que o processo de criação de \textit{drives}, bibliotecas e prototipação do dispositivo físico são 
explicados em detalhes na monografia de referência, assim como o desenvolvimento do aplicativo utilizando a linguagem de programação Java.

AINDA ESTOU ESCREVENDO ESSA PARTE
Adevan Neves Santos JAVA BACKEND

\section{Especificação}

