\ \ \\[2cm]
The undergraduate thesis presents a proposed solution to the issue of air quality monitoring 
in residential environments, named the \textbf{BURI System}. The physical prototype is designed to collect 
data on carbon monoxide concentration, relative humidity, and air temperature. The mobile application is responsible 
for visualizing the information, configuring the measurement device, and considering the monitored values to issue danger 
alerts in extreme situations, notifying users to prevent potential health risks. The development of the solution adopted an
integrated approach, with simultaneous hardware and software activities, promoting agility and autonomy in the research stages. 
Additionally, design techniques were applied to validate the prototype with undergraduate students from the State University of Amazonas. 
The results obtained from the evaluations conducted with the students were positive, highlighting the project's relevance in the Internet of Things 
field. This initiative not only democratizes access to embedded systems technologies for a broader audience but also stands out as an open-source project, 
with the potential for extension to applications in various areas of knowledge.

\noindent \textbf{Keywords}:  IoT; Android; ESP32; Embedded Systems; Carbon Monoxide.
