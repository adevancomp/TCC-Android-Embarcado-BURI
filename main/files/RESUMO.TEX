\ \ \\[2cm]
O trabalho de conclusão de curso apresenta uma proposta de solução para o problema de monitoramento da qualidade do ar em ambientes residenciais, denominada 
Sistema \textbf{BURI}. O protótipo físico tem como objetivo coletar dados sobre a concentração de monóxido de carbono, a umidade relativa e a temperatura do ar. 
Já o aplicativo móvel é responsável pela visualização das informações e pela configuração do dispositivo de medição, além de considerar os valores monitorados para 
emitir alertas de perigo em situações extremas, notificando o usuário para prevenir possíveis riscos à saúde. O desenvolvimento da solução adotou uma abordagem integrada, 
com atividades simultâneas de \textit{hardware} e \textit{software}, promovendo agilidade e autonomia na execução das etapas da pesquisa. Adicionalmente, 
foram aplicadas técnicas de \textit{design} para validar o protótipo junto a alunos de graduação da Universidade do Estado do Amazonas. Os resultados obtidos nas avaliações 
realizadas com os alunos foram positivos, evidenciando a relevância do projeto na área de Internet das Coisas. A iniciativa não apenas democratiza o acesso às tecnologias 
de sistemas embarcados para um público mais amplo, mas também se destaca por ser um projeto de código aberto, com potencial de extensão para aplicações em diversas áreas do conhecimento.



\noindent \textbf{Palavras-Chave}: IoT; Android; ESP32; Sistemas Embarcados; Monóxido de Carbono.
