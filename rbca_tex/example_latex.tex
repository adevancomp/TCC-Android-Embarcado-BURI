%%%%%%%%%%%%%%%%%%%%%%%%%%%%%%%%%%%%%%%%%%%%%%%%%%%%%%%%%%%%%
% Example of an article for the RBCA Journal                %
%                                                           %
% Please note that whilst this template provides a          %
% preview of the typeset manuscript for submission, it      %
% will not necessarily be the final publication layout.     %
%                                                           %
% a4paper: UK paper size toggle                             %
% alpha-refs: author-year citation and bibliography toggle  %
% RBCA_v3.0: RBCA latex class
%%%%%%%%%%%%%%%%%%%%%%%%%%%%%%%%%%%%%%%%%%%%%%%%%%%%%%%%%%%%%

% below we need to define the primary language of the paper - for paper in Portuguese you must change "english" by "brazilian".
\documentclass[alpha-refs,brazilian]{RBCA_v3.0}

% Uncomment this line if the article is in Portuguese.
\AtBeginDocument{\renewcommand{\harvardand}{e}}

%%% Place your packages here
%%% \usepackage[options]{package}

%%%


%%% Paper title
\title{BURI: Sistema Embarcado de monitoramento da qualidade do ar em ambiente residencial}

%%% This "second" title in English will be mandatory and will only appear in papers written in Portuguese.
%%% Este "segundo" título, em inglês, será obrigatório e aparecerá  SOMENTE em artigos escritos em português.
\titleother{BURI: Embedded System for Air Quality Monitoring in Residential Environments}

%%% Use the \authfn to add symbols for additional footnotes, if any. 1 is reserved for correspondence e-mails; then continuing with 2,... for contributions.
%%% The orcid number of authors (\orcid) is optional.
\author[1]{Adevan Neves Santos \orcid{0009-0006-2716-5958}}
\author[1]{Jose Renato Satiro Santiago Junior \orcid{0000-0000-0000-0000}}
%\author[2]{Third Author}
%\author[2]{Fourth Author}

\affil[1]{Universidade do Estado do Amazonas}
%\affil[2]{Second Institution}

%%% Author e-mails
\authnote{\authfn{1}ans.eng20@uea.edu.br; jrssjunior@uea.edu.br}

%%% "Short" author for running page header - until 3 authors
\runningauthor{Adevan Neves Santos \& Jose Renato Satiro Santiago Junior}

%%% "Short" author for running page header - more than 3 authors
%\runningauthor{First et al.}    

%%% Paper category:
%%% English  : Original paper,  Experience Report     or Tutorial
%%% Português: Artigo original, Relato de experiência ou Tutorial
\papercat{Artigo original}

%%% Editor should only set information about the issue journal and the paper
\jvolume{11}           % volume number
\jissue{3}             % issue number
\jyear{2019}           % edition year
\jmonth{November}      % edition month

\setcounter{page}{1}   % number of the first page of the paper
\firstpage{1}
\jid{9999}             % paper ID 

\jrec{yyyy-mm-dd}      % date of article submission
\jrev{yyyy-mm-dd}      % date of final paper revision
\jacc{yyyy-mm-dd}      % date of paper accepted


\begin{document}

\begin{frontmatter}
	
\maketitle
\thispagestyle{empty}

\begin{Abstract} % abstract in english
	Carbon monoxide (CO) is a toxic, colorless, and odorless gas with a high asphyxiating capacity. By chemically bonding with hemoglobin, the protein in blood responsible for oxygen transport, with an affinity 200 
	to 300 times greater than that of oxygen, this substance reduces cellular respiration and can lead to death in cases of prolonged exposure. In this context, the BURI system emerges as a solution proposal based 
	on an embedded system for air quality monitoring, focusing on data collection to identify risk situations. The system provides real-time information to the user and supports the adoption of preventive measures. 
	The developed prototype was evaluated by undergraduate engineering students, who reported positive results regarding the use of the application and system configuration.
\end{Abstract}

\begin{keywords}
	Android; ESP32; IoT; Carbon Monoxide
\end{keywords}

\begin{resumo} % resumo em português
	O monóxido de carbono (CO) é um gás tóxico, incolor e inodoro, com elevada capacidade asfixiante. Ao estabelecer ligação química com a hemoglobina, proteína presente no sangue responsável pelo transporte de oxigênio, 
	com uma afinidade de 200 a 300 vezes superior à do oxigênio, essa substância reduz a respiração celular e pode levar ao óbito em casos de exposição prolongada. Nesse contexto, o sistema BURI apresenta-se como uma proposta 
	de solução baseada em um sistema embarcado para o monitoramento da qualidade do ar, com foco na coleta de dados para identificar situações de risco. O sistema fornece ao usuário informações em tempo real e apoia a adoção de 
	medidas preventivas. O protótipo desenvolvido foi avaliado por alunos graduandos de engenharia, que relataram bons resultados em relação ao uso do aplicativo e à configuração do sistema.
\end{resumo}

\begin{palavras_chave} % palavras-chave em português 
	Android; ESP32; IoT; Monóxido de Carbono
\end{palavras_chave}

\end{frontmatter}

\section{Introdução}

%% Specify your .bib file name here, without the extension .bib
\bibliography{references} 

\end{document}
