%%%%%%%%%%%%%%%%%%%%%%%%%%%%%%%%%%%%%%%%%%%%%%%%%%%%%%%%%%%%%
% Example of an article for the RBCA Journal                %
%                                                           %
% Please note that whilst this template provides a          %
% preview of the typeset manuscript for submission, it      %
% will not necessarily be the final publication layout.     %
%                                                           %
% a4paper: UK paper size toggle                             %
% alpha-refs: author-year citation and bibliography toggle  %
% RBCA_v3.0: RBCA latex class
%%%%%%%%%%%%%%%%%%%%%%%%%%%%%%%%%%%%%%%%%%%%%%%%%%%%%%%%%%%%%

% below we need to define the primary language of the paper - for paper in Portuguese you must change "english" by "brazilian".
\documentclass[alpha-refs,brazilian]{RBCA_v3.0}

% Uncomment this line if the article is in Portuguese.
\AtBeginDocument{\renewcommand{\harvardand}{e}}

%%% Place your packages here
%%% \usepackage[options]{package}

%%%


%%% Paper title
\title{BURI: Sistema Embarcado de monitoramento da qualidade do ar em ambiente residencial}

%%% This "second" title in English will be mandatory and will only appear in papers written in Portuguese.
%%% Este "segundo" título, em inglês, será obrigatório e aparecerá  SOMENTE em artigos escritos em português.
\titleother{BURI: Embedded System for Air Quality Monitoring in Residential Environments}

%%% Use the \authfn to add symbols for additional footnotes, if any. 1 is reserved for correspondence e-mails; then continuing with 2,... for contributions.
%%% The orcid number of authors (\orcid) is optional.
\author[1]{Adevan Neves Santos \orcid{0009-0006-2716-5958}}
\author[1]{Jose Renato Satiro Santiago Junior \orcid{0000-0000-0000-0000}}
%\author[2]{Third Author}
%\author[2]{Fourth Author}

\affil[1]{Universidade do Estado do Amazonas (UEA)}
%\affil[2]{Second Institution}

%%% Author e-mails
\authnote{\authfn{1}ans.eng20@uea.edu.br; jrssjunior@uea.edu.br}

%%% "Short" author for running page header - until 3 authors
\runningauthor{Adevan Neves Santos \& Jose Renato Satiro Santiago Junior}

%%% "Short" author for running page header - more than 3 authors
%\runningauthor{First et al.}    

%%% Paper category:
%%% English  : Original paper,  Experience Report     or Tutorial
%%% Português: Artigo original, Relato de experiência ou Tutorial
\papercat{Artigo original}

%%% Editor should only set information about the issue journal and the paper
\jvolume{11}           % volume number
\jissue{3}             % issue number
\jyear{2019}           % edition year
\jmonth{November}      % edition month

\setcounter{page}{1}   % number of the first page of the paper
\firstpage{1}
\jid{9999}             % paper ID 

\jrec{yyyy-mm-dd}      % date of article submission
\jrev{yyyy-mm-dd}      % date of final paper revision
\jacc{yyyy-mm-dd}      % date of paper accepted


\begin{document}

\begin{frontmatter}
	
\maketitle
\thispagestyle{empty}

\begin{Abstract} % abstract in english
	Carbon monoxide (CO) is a toxic, colorless, and odorless gas recognized for its high asphyxiating capacity. It forms a chemical bond with hemoglobin, a blood protein responsible for oxygen transport, with an affinity 
	200 to 300 times greater than that of oxygen, impairing the process of cellular respiration. This characteristic can cause severe damage to the body and lead to death in cases of prolonged exposure or high concentrations. 
	In light of this issue, the BURI system emerges as a proposed solution for air quality monitoring. Based on an embedded system, BURI collects sensor data and identifies risk situations, providing real-time information to users. 
	Additionally, it uses resilient communication protocols that work with or without internet access, issuing alerts directly to the smartphone, allowing for preventive measures to be taken. The prototype was evaluated in practice 
	by engineering students, who highlighted the system's efficiency in providing data and simplified setup process. These results reinforce the viability of the proposed solution to mitigate the risks associated with carbon monoxide 
	exposure and promote safer environments.
\end{Abstract}

\begin{keywords}
	Android; ESP32; IoT; Carbon Monoxide
\end{keywords}

\begin{resumo} % resumo em português
	O monóxido de carbono (CO) é um gás tóxico, incolor e inodoro, reconhecido por sua elevada capacidade asfixiante. Ele forma uma ligação química com a hemoglobina, proteína do sangue responsável pelo transporte 
	de oxigênio, com uma afinidade 200 a 300 vezes maior que a do gás oxigênio, comprometendo o processo de respiração celular. Essa característica pode provocar danos graves ao organismo e levar ao óbito em casos 
	de exposição prolongada ou concentrações elevadas. Diante desse problema, o sistema BURI surge como uma proposta de solução para o monitoramento da qualidade do ar. Baseado em um sistema embarcado, o BURI coleta 
	dados de sensores e identifica situações de risco, oferecendo informações em tempo real aos usuários. Além disso, utiliza protocolos de comunicação resilientes, que funcionam com ou sem acesso à internet, e emite 
	alerta diretamente no \textit{smartphone}, permitindo a adoção de medidas preventivas. O protótipo foi submetido à avaliação prática por graduandos de engenharia, que destacaram a eficiência do sistema em fornecer 
	dados e processo simplificado de configuração. Tais resultados reforçam a viabilidade da solução proposta para mitigar os riscos associados à exposição ao monóxido de carbono e promover ambientes mais seguros.
\end{resumo}

\begin{palavras_chave} % palavras-chave em português 
	Android; ESP32; IoT; Monóxido de Carbono
\end{palavras_chave}

\end{frontmatter}

\section{Introdução}

O monóxido de carbono (CO) é uma substância química gerada pela combustão incompleta de materiais ricos em carbono, em condições de baixo oxigênio. 
Embora o CO seja um componente químico produzido pelos processos fisiológicos normais do organismo, sua inalação em quantidades elevadas aumenta o 
risco de intoxicação. Ao ser inalado, o gás difunde-se nos alvéolos pulmonares e segue para a corrente sanguínea, onde se liga à proteína hemoglobina, 
que atua no transporte de oxigênio no sangue, interrompendo o processo adequado de respiração celular no indivíduo \citep{gases-sufocantes-hernandez-2022}.


%% Specify your .bib file name here, without the extension .bib
\bibliography{references} 

\end{document}
