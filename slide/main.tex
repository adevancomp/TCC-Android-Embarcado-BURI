\documentclass[12pt]{beamer}
\usetheme{Madrid}
\usepackage[utf8]{inputenc}
\usepackage[portuguese]{babel}
\usepackage[T1]{fontenc}
\usepackage{amsmath}
\usepackage{amsfonts}
\usepackage{amssymb}
\usepackage{graphicx}
\usepackage{capt-of}
\usepackage{tabularx}
\setbeamertemplate{navigation symbols}{} 
\colorlet{beamer@blendedblue}{green!60!black}
\logo{\includegraphics[scale=0.03]{./img/uea-new.png}}
\usepackage[backend=biber]{biblatex}
\addbibresource{references.bib}

%------------- CAPA ----------------

\title{BURI}
\subtitle{Sistema Embarcado de monitoramento da qualidade do ar em ambiente residencial}
\author{Adevan Neves Santos \\ Orientador: Prof. Jonathas Silva dos Santos}

\date{12 de dezembro de 2024}

\setbeamertemplate{footline}{%
  \leavevmode%
  \hbox{%
    \begin{beamercolorbox}[wd=.33\paperwidth,ht=2.5ex,dp=1ex,center,leftskip=3mm]{author in head/foot}%
      \usebeamerfont{author in head/foot}Adevan Neves Santos
    \end{beamercolorbox}%
    \begin{beamercolorbox}[wd=.34\paperwidth,ht=2.5ex,dp=1ex,center]{title in head/foot}%
      \usebeamerfont{title in head/foot}\insertframenumber/\inserttotalframenumber
    \end{beamercolorbox}%
    \begin{beamercolorbox}[wd=.33\paperwidth,ht=2.5ex,dp=1ex,center,rightskip=3mm]{date in head/foot}%
      \usebeamerfont{date in head/foot}BURI
    \end{beamercolorbox}%
  }%
  \vskip0pt%
}

%------------- CAPA ----------------

% As dicas de cada seção foram retiradas do vídeo: https://www.instagram.com/p/DCCn51VxLwa/

\begin{document}
    \maketitle

    \begin{frame}
        \frametitle{Sumário}
        \tableofcontents
    \end{frame}

    \section{Introdução}
    %Apresente o título do TCC e introduza rapidamente o contexto
    %pergunta de pesquisa e os objetivos

    \begin{frame}{Contextualização}
        O monóxido de carbono (CO) é um gás incolor 
    \end{frame}

    \section{Justificativa}
    %Explique a relevância do trabalho de forma sucinta, justificam
    %sua importância acadêmica ou prática.

    \begin{frame}
        O monóxido de carbono (CO) é um gás incolor 
    \end{frame}

    \section{Referencial Teórico}
    %Destaque os principais conceitos e autores que embasam o
    %trabalho, sem entrar em detalhes excessivos.

    \begin{frame}
        O monóxido de carbono (CO) é um gás incolor 
    \end{frame}

    \section{Metodologia}
    %Resuma a metodologia utilizada, incluindo o tipo de pesquisa,
    %métodos de coleta e análise de dados

    \begin{frame}
        O monóxido de carbono (CO) é um gás incolor 
    \end{frame}

    \section{Resultados}
    %Foque nos resultados mais relevantes e significativos,
    %destacando como eles respondem à pergunta de pesquisa.

    \begin{frame}
        O monóxido de carbono (CO) é um gás incolor 
    \end{frame}

    \section{Discussão}
    %Relacione os resultados com o referencial teórico e com outros
    %estudos, interpretando o que eles significam

    \begin{frame}
        O monóxido de carbono (CO) é um gás incolor 
    \end{frame}

    \section{Considerações Finais}
    %Enfatize as conclusões e implicações principais, 
    %incluindo sugestões para pesquisas futuras.

    \begin{frame}
        O monóxido de carbono (CO) é um gás incolor 
    \end{frame}

    \section{Referências}
    %Mencione que as referências estão devidamente listadas no 
    %trabalho, sem detalhá-las.
    
    \printbibliography
\end{document}